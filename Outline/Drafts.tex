\section*{Table of Content}
\subsection*{Proposal 1}

\begin{enumerate}[font=\bfseries]
    \item \textbf{Introduction}
    \item \textbf{The Dynamic HIV Model}
    \begin{itemize}[font=\small]
        \item \textit{(short) Model review}: Linear/nonlinear; Different models simulate different interactions, 
        cell popultaions, virus developments (e. g. resistance) etc.
        \item \textit{Model requirements}: What should our model be capable of simulating?
        \item \textit{Model definition/selection}: Introduce a modified version of the model in \cite{wu2010game}; 
        Which modifications and why?; Introduce all parameters (biological meaning + reference values \cite{adams2005hiv} + units) and state 
        which have to be estimated from patient's data \cite{callaway2002hiv}
    \end{itemize}
    \item \textbf{The Clinical Data}
    \begin{itemize}[font=\small]
        \item \textit{Numerical simulation}: Predefine a set of true parameters, including initial drug dosages; Interprete a set of numerical results as a sequence of blood 
        tests (e.g. every other day within the first 8 weeks)
    \end{itemize}
    \item \textbf{The Methods to Personalize the Model}
    \begin{enumerate}[font=\bfseries]
        \item \textbf{Probabilistic Formulation of Prior Knowledge and Bayes' Theorem}
        \begin{itemize}[font=\small]
            \item \textit{Prior Knowledge in Data Space}: clinical data suffers from measurement errors (e.g. environment of blood tests can 
            not accurately be reconstructed, different time intervals elapsed between blood test and evaluation, uncertainties in instruments 
            used to evaluate viral load etc.); blood tests are independent samples; cells counts can only be positive and change fast in the 
            first weeks of the treatment
            \item \textit{Prior Knowledge in Data Space}: Any biological contraints on parameters? Is there a reference value and how strong
            are the expected deviations from it?
            \item \textit{Bayes' Theorem}
        \end{itemize}
        \item \textbf{Sampling with Markov Chain Monte Carlo Method}
        \begin{itemize}[font=\small]
            \item \textit{Implementation Details}
        \end{itemize}
    \end{enumerate}
    \item \textbf{Results and Discussion}
    \item \textbf{Summary and Outlook}
    \begin{itemize}[font=\small]
        \item \textit{Outlook}: this personalized dynamic model can be used to solve an optimal control problem to find ideal drug dosages
    \end{itemize}
\end{enumerate}

\subsection*{Proposal 2}
More comprehensive alternative: the focus might not only be on the inversion but the additional sections for Optimal Control won't 
go into detail and only introduce the broad idea


\begin{enumerate}[font=\bfseries]
    \item \textbf{Introduction}
    \item \textbf{The Model for a Personalized HIV Treatment}
    \begin{enumerate}[font=\bfseries]
        \item \textbf{The Dynamic Model}
        \begin{itemize}[font=\small]
            \item \textit{see Proposal 1}
        \end{itemize}
        \item \textbf{The Optimal Control Problem}
        \begin{itemize}[font=\small]
            \item \textit{Control Parameter}: drug dosages
            \item \textit{Cost function}: define the cost function which has to be maximized in order to minimize the side effects
        \end{itemize}
    \end{enumerate}
    \item \textbf{The Clinical Data}
    \begin{itemize}[font=\small]
        \item \textit{see Proposal 1}
    \end{itemize}
    \item \textbf{The Methods}
    \begin{enumerate}[font=\bfseries]
        \item \textbf{Personalized Model - Bayesian Approach for Parameter Estimation}
        \begin{itemize}[font=\small]
            \item \textit{see Methods section of Proposal 1}
        \end{itemize}
        \begin{enumerate}[font=\bfseries]
            \item \textbf{Probabilistic Formulation of Prior Knowledge and Bayes' Theorem}
            \item \textbf{Sampling with Markov Chain Monte Carlo Methods}
        \end{enumerate}
        \item \textbf{Adjusted drug dose - Optimal control}
        \begin{itemize}[font=\small]
            \item \textit{Computing optimal controls}: state idea of Pontryagin's Maximum Principle
        \end{itemize}
    \end{enumerate}
    \item \textbf{Results and Discussion}
    \begin{itemize}[font=\small]
        \item focus on parameter estimation
    \end{itemize}
    \item \textbf{Summary and Outlook}
\end{enumerate}