\section{Introduction}
\label{sec:intro}

From its first occurrence in the early beginning of the 1980s until this very day, 
\textit{Human Immunodeficiency Viruses}, widely known as HIV, have taken 39 million lives.
The brutalitiy of the pandemic has motivated scientists all over the world to intensively study the
pathogenesis and life cycle of the virus.\par

Nowadays, it is well-known that HIV directly affacts the patient’s adaptive immunity.
Under healthy circumstances, this component of the human immune system fights pathogens by a 
conglomerate of communicating cells with so-called CD4+ or T helper cells as key players.
Taking the role of coordinators, they trigger the crucial steps in the immune response which are 
necesseary to eliminate free virus and already infected cells \cite{alberts2002helper}.\newline
In this highly sensitive procedure, HIV attacks the body.
Among other target cells, the virus infects CD4+ T lymphocytes and hence paralyses the control center of the adaptive immune
system.
Consequently, the antibody secretion, originally stimulated by the T helper cells as well as the immunity, that is mediated by cells,
break down \cite{buselmaier2018biologie}.\par

The course of an untreated HIV infection differs from patient to patient and so do the observable symptoms, 
if they appear at all.
More reliable metrics to track the extent and stage of the viremia, are the viral load and CD4+ cell 
count in the patient's blood.
After the number of viruses initially exhibits a dramatic increase, causing flu-like symptoms, it stabilizes at some lower level.
This, mostly several years lasting stage of clinical latency, is mainly free of symptoms.
In spite of this deceptive silence, the virus remains in the body and continously lowers the CD4+ T-cell count.
At a certain point, the viral load rises rapidly, reducing the T-cell count to a level at which it cannot provide 
sufficent protection any more.
In this stage, defined as \textit{acquired immunodeficiency syndrome} or shorter AIDS, patients suffer more frequentely 
and more seriously from opportunistic infections and even cancer.
Finally, one of these secondary infections ends fatally \cite{mittler1998influence}.\par

Until the mid-90's, an HIV infected person was most likely doomed to die.
Relief from this devastating disease first came with the invention of so-called antiretroviral medication.
These drugs interfere with the continously better understood multistep replicative life cycle of HIV in 
human CD4+ T cells, aiming on halting or at least slowing it down.
Patients under \textit{Highly Active Antiretrovial Therapies} (HAARTs) achieve long-lasting viral suppression by 
combining multiple of such antiretroviral agents.
Although these treatment strategies are very efficient in reducing morbidity and mortality of an HIV infection, a
full viral eradication remains elusive \cite{pau2014antiretroviral,simon2006hiv}.
A final cure is inhibited by two factors.
Firstly, although the stabilized virus counts even fall below the threshold of detecability of most assays, 
it has been discovered that low levels of free virus remain in the plasma.
In addition to this residual viremia, infected memory CD4+ T cells and macrophages as well as dendritic cells that bounded 
free virus, constitute latent virus reservoirs.
These, most of the time silent cells, resume to viral reproduction as soon as the host cell is reactivated.
A trigger herefore, could for instance be an interrupted HAART \cite{ruelas2013integrated,shen2008viral}.\par

This observation demonstrates an essential drawback of antiretroviral therapies. Deviations from the strict and 
complex dosing schedule can not only lead to drug resistance mutations but can also cause a viral rebound.
Thus, a full compliance of the patient - meaning the intake of a daily cocktail of pills for the rest of her or his life - is inevitable.
In addition hereto, antiretroviral medication also comes along with unpleasant side effects, ranging from low-grade intolerances to 
life-threatening reactions.
All in all, such drug toxicities in combination with a lifelong commitment, adherence issues, the high possibility of drug resistance, 
high costs and life style issues, make it even harder for the patient to fully commit to the treatment and increase the pressure on science
to further improve the HIV therapies \cite{lu2018haart,pau2014antiretroviral}.\par

Remedy can not only be provided by developing novel pharmaceuticals, but also by correctly using the currently available medicine.
Due to the diversity of genetic and molecular conditions in different HIV infected patients, an ultimately best therapy does 
not exist. 
Wrong combinations of antiretroviral drugs or inappropriate dosing schedules can amplify the side effects and downsides of the current 
therapeutical standards.
Hence, instead of prescribing an off-the-shelf therapy, medical treatment should be tailored to each patient individually, taking her or 
his anamnesis and personal characteristics into account \cite{lu2018haart}.\newline
This idea however, prompts the justified question of how to find the optimal medication.
Testing different treatments by trial and error is cumbersome and painful.
Exposing patients to this additional psychological and physical burden is unacceptable.\newline
A promising alternative to this \textit{in vivo} approach is a systematic \textit{in silicon} methodology.
Here, the human body is interpreted as a dynamic system and described by mathematical models \cite{rosenberg2007using}.
The free design parameters of an HIV therapy as well as the characteristics of the considered patient are then encoded by a set of mathematical 
parameters.
Applying tools of inverse theory to estimate the latters from clinical data us to adjust the model to each patient individually.
The resulting personalized dynamic system is subsequently used to formulate a control problem.
A cost function quantifies not only the success of a therapy but also its burdens which are controlled by parameters such as drug dosages.
Minimizing this functional, allows to determine the individually, ideal HIV therapy in a short time.\par

The following chapter aims at introducing a data-driven workflow that determines an individually optimized HIV treatment.
For this, we define an appropriate HIV model and state the control problem, which both provide the foundation of the methodology.
Following, the clinical data of a specific patient is simulated by numerically solving the dynamic system.
The fourth section demonstrates how we exploit this data to personalize the HIV model by using a Bayesian approach in combination with 
\textit{Markov Chain Monte Carlo} (MCMC) sampling algorithms.
Techniques to solve the arising optimal control problem are mentioned but not explained in detail.
Finally, we discuss results, future ideas and potential improvements.\newline