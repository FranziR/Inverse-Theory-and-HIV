\section{Discussion and Conclusion}
\label{sec:Dis}

A crucial aspect of the success of the introduced approach is the reliability of the observations $V_{obs}$.
To diminish measurement errors and reduce the effect of outliers one could extract multiple samples of plasma every other week and average the viral loads.\\
In addition hereto, the prior distribution in model space $p(\mathbf{d}_{obs}|\mathbf{m})$ has inconsistencies relating to the independence of the observations.
As already discussed in chapter \ref{sec:Data}, dependent on the value of each $V_{obs}(t_i)$, the joint probability distribution might be set to 0.
At the same time, if each $V_{obs}(t_i) > 0$, the measurements are assumed to be independent.
Hence, different prior probability distributions should be investigated.\\
Generally, the prior distributions in model and data space are heavily subjective.
In order to perfectly model the prior knowledge it would be reasonable to analyse the choices of the variance $\sigma_D^2$ and $Var[\mathbf{m}]$ as well as the distributions themselves.\\ \\
Although, some aspects of the introduced methodology still leave freedom for improvement, the presented work lays the foundation for an individually tailored drug treatment and to find the optimal drug adjustment for each patient.