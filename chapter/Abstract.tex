\begin{abstract}

During the last 40 years, \textit{Human Immunodeficiency Viruses}, short HIV, has been a permanent companion of human mankind.
In the recent past, mortality rate could be decreased thanks to highly developed drugs.
However, still patients might suffer from strong side effects which again lower their life quality.
This report aims on developing a methodology to individually tailor kind and dose of antiretroviral treatment to each patient.
In a first step, an approach how to quantify the efficiency of a drug is introduced.
By defining a set of ODEs this metric can be indirectly correlated to observable viral load within the blood plasma of the patient.
In order to infer statements about the drug's efficiency based on observations, the problem is formulated as a Bayes inversion problem.
Prior knowledge about the model and observations is represented by probability distributions which can be used to find a non-explicit definition of the posterior probability distribution of the model.
Using a Simulated Annealing algorithm allows to sample from this posterior distribution in order to find the model with maximal posterior likelihood.
Based on this, the efficiency of the drug is computed and the therapy either accepted or rejected and a new medication tried.

\end{abstract}