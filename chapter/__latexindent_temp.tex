\section{Introduction}
\label{sec:Introduction}

\begin{itemize}
    \item attantion getter/intro
    \item Since then, pathogenesis of HIV has been studied and the life cycle of the virus no well-known
    \item describe it 
        --> parts of human immune system is based on a network of different cells \dots
        --> under normal conditions, T helper cells for instance recognize harmful antigenes and activate their companions, 
            the T killer cells, B cells and marcopharges
        --> these then finally kill the intruder, create antibodies \dots
        --> HIV breaks this chain of protection by attacing the CD4+ T helper cells
    \item knowledge about the infection mechanisms of HIV allows the development of specfic drugs
    \item meanwhile, several different so-called antiretroviral medications available
    \item 
    \item 
\end{itemize}

\textbf{Replication cycle of HIV virion}
\begin{itemize}
    \item for replication, the HIV virion requires a host cell which it abuses to multiple
    \item HIV can infect a variety of different immune cells, though the most linkely ones are CD4+ T helper cells and macrophages
    \item the replication cycle startes with the entry of the virion  into its designated host cell
    \item once entered, the genome encoding RNA of the virion is copied into a complementary DNA by the enzyme reverse transcriptase
    \item the viral DNA can then be integrated into the host's genome
    \item the process of transcription is extremly error-prone and the resulting mutations may cause drug resistance
    \item the integrated viral DNA can lay dormaint during the 
\end{itemize}

\textbf{Normally working immune system}
\begin{itemize}
    \item HIV is a disease which attacs the human immune system
    \item human immune system is highly complex and comprises from multiple, co-dependent components
    \item one of these, is the adaptive immune response which again is subdivided into two types
    \item the cell-mediated immune response is carried out by T cells, while the humoral immune response
     is controlled by activated B cells and antibodies
    \item the memory of the adaptive immune response allows a fast and efficient protection from reinfection with the same pathogen
    \item 
    \item Helper T cells are arguably the most important cells in adaptive immunity, 
    as they are required for almost all adaptive immune responses
    \item ein Bestandteil davon, sind die zellulären Abwehrreaktoin welche beispielsweise 
    dafür zuständig sind, bösartige Viren 
    \item die zelluläre Immunreaktion wid von den sogenannten T Helfer Zellen kooridnert 
\end{itemize}



From its first occurrence in the early beginning of the 1980s until this very day, \textit{Human Immunodeficiency Viruses}, widely known as HIV, has taken 39 million lives.
Thanks to enhanced prevention and highly sophisticated medication, the brutality of the pandemic has been remarkably reduced since its peak in 2004.
Although, nowadays numerous antiretroviral treatment exist, aiming on inhibiting the viruses, patients suffers from strong side effects.
Optimally, the choice of kind and dose of drug has to be made individually for each patient.
However, finding the perfect medication by trial and error is cumbersome and painful.\\
This scientific report is not only concerned with finding a metric to quantify the efficiency of HIV treatments but also develops a systematic approach how to derive this efficiency measure individually for each patient.\\ \\
% Nevertheless, for a profound understanding of the HIV infection and its antiviral treatment on a long-term scale, further HIV studies are necessary.
For this, Adames et. al \cite{ADAMS200510} suggest to model the dynamics of the viruses under the effect of an antiretroviral treatment by the following set of non-linear ODEs (\textit{Ordinary Differential Equations}): 

\begin{align}
    \begin{split}
        \dot{T} &= \lambda - \rho T - (1 - \gamma(t))kTV\\
        \dot{T}^{*} &= (1-\gamma(t))kTV-\delta T^{*}\\
        \dot{V} &= N\delta T^{*}-cV \quad \text{.}
    \end{split}
    \label{equ:ODEs}
\end{align}

Although, HIV has more than one potential target cell\footnote{Target cells are cells that are infected by HIV, mainly the helper T cells such as CD4+ T cells or macrophages.} the above model considers only the population of one, denoted by $T$.
$T^*$ describe the infected target cells which actively produces free viruses $V$. 
Further $\lambda$ in $\frac{cells}{day\cdot ml}$ represents the rate at which new $T$ cells are created from sources within the body, $\rho$ in $\frac{1}{day}$ is the death rate of $T$ cells, $k$ in $\frac{ml}{virions\cdot day}$ the rate by which $T$ cells are infected by virus, $\delta$ in $\frac{1}{day}$ is the death rate for infected cells , $N$ the number of new virions produced from each of the infected cells during their lifetime, and finally $c$ in $\frac{1}{day}$ is the clearance rate of free virions. 
The time-varying parameter $\gamma(t) \in [0,1]$ quantifies the antiviral drug efficacy whose formal definition will not be given here.
A drug is said to be efficient if all viruses are inhibited, i.e. $\gamma(t) = 1$.
Only if this is the case, the ODEs \ref{equ:ODEs} can be solved analytically, otherwise numerically.
In the following, we will assume an imperfect inhibition and denote the numerical solution of the viral load by $\tilde{V}_j \approx V(t_j)$ at time $t_j$ (the considered time period is discretized in $N_t$ time steps).
Further, Huang et. al claim that if $\gamma(t) > e_c$ for all $t$ the virus will be eventually eradicated.
Thus, the so-called \textit{efficacy threshold} $e_c$, defined as

\begin{align}
 e_c = 1-\frac{c\rho}{kN}\lambda
 \label{equ:e_c}
\end{align}

can be interpreted as the ability of a patient’s immune system to control viral replication.
For an optimized drug treatment it is important to know $e_c$ for each patient.
At the same time, only the viral load $V(t)$ can directly be measured in the blood plasma.
Hence, $e_c$ can only be retrieved by estimating the model parameters $\mathbf{m} = (c, \rho, k, N, \lambda)$ from the patient's clinical data  \cite{huang2003modeling}.\\
All in all, estimating the patient specific, discrete model parameter $\mathbf{m}$ is an inverse problem.
The associated forward problem is given by the set of equations \ref{equ:ODEs} which non-linearly relates the model parameter $\mathbf{m}$ to the observations $V_{obs}$.